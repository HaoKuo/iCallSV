% Generated by Sphinx.
\def\sphinxdocclass{report}
\documentclass[letterpaper,10pt,english]{sphinxmanual}
\usepackage[utf8]{inputenc}
\DeclareUnicodeCharacter{00A0}{\nobreakspace}
\usepackage{cmap}
\usepackage[T1]{fontenc}
\usepackage{babel}
\usepackage{times}
\usepackage[Sonny]{fncychap}
\usepackage{longtable}
\usepackage{sphinx}
\usepackage{multirow}
\usepackage{eqparbox}


\addto\captionsenglish{\renewcommand{\figurename}{Fig. }}
\addto\captionsenglish{\renewcommand{\tablename}{Table }}
\SetupFloatingEnvironment{literal-block}{name=Listing }



\title{iCallSV Documentation}
\date{June 04, 2016}
\release{}
\author{Author}
\newcommand{\sphinxlogo}{}
\renewcommand{\releasename}{Release}
\setcounter{tocdepth}{3}
\makeindex

\makeatletter
\def\PYG@reset{\let\PYG@it=\relax \let\PYG@bf=\relax%
    \let\PYG@ul=\relax \let\PYG@tc=\relax%
    \let\PYG@bc=\relax \let\PYG@ff=\relax}
\def\PYG@tok#1{\csname PYG@tok@#1\endcsname}
\def\PYG@toks#1+{\ifx\relax#1\empty\else%
    \PYG@tok{#1}\expandafter\PYG@toks\fi}
\def\PYG@do#1{\PYG@bc{\PYG@tc{\PYG@ul{%
    \PYG@it{\PYG@bf{\PYG@ff{#1}}}}}}}
\def\PYG#1#2{\PYG@reset\PYG@toks#1+\relax+\PYG@do{#2}}

\expandafter\def\csname PYG@tok@gd\endcsname{\def\PYG@tc##1{\textcolor[rgb]{0.63,0.00,0.00}{##1}}}
\expandafter\def\csname PYG@tok@gu\endcsname{\let\PYG@bf=\textbf\def\PYG@tc##1{\textcolor[rgb]{0.50,0.00,0.50}{##1}}}
\expandafter\def\csname PYG@tok@gt\endcsname{\def\PYG@tc##1{\textcolor[rgb]{0.00,0.27,0.87}{##1}}}
\expandafter\def\csname PYG@tok@gs\endcsname{\let\PYG@bf=\textbf}
\expandafter\def\csname PYG@tok@gr\endcsname{\def\PYG@tc##1{\textcolor[rgb]{1.00,0.00,0.00}{##1}}}
\expandafter\def\csname PYG@tok@cm\endcsname{\let\PYG@it=\textit\def\PYG@tc##1{\textcolor[rgb]{0.25,0.50,0.56}{##1}}}
\expandafter\def\csname PYG@tok@vg\endcsname{\def\PYG@tc##1{\textcolor[rgb]{0.73,0.38,0.84}{##1}}}
\expandafter\def\csname PYG@tok@vi\endcsname{\def\PYG@tc##1{\textcolor[rgb]{0.73,0.38,0.84}{##1}}}
\expandafter\def\csname PYG@tok@mh\endcsname{\def\PYG@tc##1{\textcolor[rgb]{0.13,0.50,0.31}{##1}}}
\expandafter\def\csname PYG@tok@cs\endcsname{\def\PYG@tc##1{\textcolor[rgb]{0.25,0.50,0.56}{##1}}\def\PYG@bc##1{\setlength{\fboxsep}{0pt}\colorbox[rgb]{1.00,0.94,0.94}{\strut ##1}}}
\expandafter\def\csname PYG@tok@ge\endcsname{\let\PYG@it=\textit}
\expandafter\def\csname PYG@tok@vc\endcsname{\def\PYG@tc##1{\textcolor[rgb]{0.73,0.38,0.84}{##1}}}
\expandafter\def\csname PYG@tok@il\endcsname{\def\PYG@tc##1{\textcolor[rgb]{0.13,0.50,0.31}{##1}}}
\expandafter\def\csname PYG@tok@go\endcsname{\def\PYG@tc##1{\textcolor[rgb]{0.20,0.20,0.20}{##1}}}
\expandafter\def\csname PYG@tok@cp\endcsname{\def\PYG@tc##1{\textcolor[rgb]{0.00,0.44,0.13}{##1}}}
\expandafter\def\csname PYG@tok@gi\endcsname{\def\PYG@tc##1{\textcolor[rgb]{0.00,0.63,0.00}{##1}}}
\expandafter\def\csname PYG@tok@gh\endcsname{\let\PYG@bf=\textbf\def\PYG@tc##1{\textcolor[rgb]{0.00,0.00,0.50}{##1}}}
\expandafter\def\csname PYG@tok@ni\endcsname{\let\PYG@bf=\textbf\def\PYG@tc##1{\textcolor[rgb]{0.84,0.33,0.22}{##1}}}
\expandafter\def\csname PYG@tok@nl\endcsname{\let\PYG@bf=\textbf\def\PYG@tc##1{\textcolor[rgb]{0.00,0.13,0.44}{##1}}}
\expandafter\def\csname PYG@tok@nn\endcsname{\let\PYG@bf=\textbf\def\PYG@tc##1{\textcolor[rgb]{0.05,0.52,0.71}{##1}}}
\expandafter\def\csname PYG@tok@no\endcsname{\def\PYG@tc##1{\textcolor[rgb]{0.38,0.68,0.84}{##1}}}
\expandafter\def\csname PYG@tok@na\endcsname{\def\PYG@tc##1{\textcolor[rgb]{0.25,0.44,0.63}{##1}}}
\expandafter\def\csname PYG@tok@nb\endcsname{\def\PYG@tc##1{\textcolor[rgb]{0.00,0.44,0.13}{##1}}}
\expandafter\def\csname PYG@tok@nc\endcsname{\let\PYG@bf=\textbf\def\PYG@tc##1{\textcolor[rgb]{0.05,0.52,0.71}{##1}}}
\expandafter\def\csname PYG@tok@nd\endcsname{\let\PYG@bf=\textbf\def\PYG@tc##1{\textcolor[rgb]{0.33,0.33,0.33}{##1}}}
\expandafter\def\csname PYG@tok@ne\endcsname{\def\PYG@tc##1{\textcolor[rgb]{0.00,0.44,0.13}{##1}}}
\expandafter\def\csname PYG@tok@nf\endcsname{\def\PYG@tc##1{\textcolor[rgb]{0.02,0.16,0.49}{##1}}}
\expandafter\def\csname PYG@tok@si\endcsname{\let\PYG@it=\textit\def\PYG@tc##1{\textcolor[rgb]{0.44,0.63,0.82}{##1}}}
\expandafter\def\csname PYG@tok@s2\endcsname{\def\PYG@tc##1{\textcolor[rgb]{0.25,0.44,0.63}{##1}}}
\expandafter\def\csname PYG@tok@nt\endcsname{\let\PYG@bf=\textbf\def\PYG@tc##1{\textcolor[rgb]{0.02,0.16,0.45}{##1}}}
\expandafter\def\csname PYG@tok@nv\endcsname{\def\PYG@tc##1{\textcolor[rgb]{0.73,0.38,0.84}{##1}}}
\expandafter\def\csname PYG@tok@s1\endcsname{\def\PYG@tc##1{\textcolor[rgb]{0.25,0.44,0.63}{##1}}}
\expandafter\def\csname PYG@tok@ch\endcsname{\let\PYG@it=\textit\def\PYG@tc##1{\textcolor[rgb]{0.25,0.50,0.56}{##1}}}
\expandafter\def\csname PYG@tok@m\endcsname{\def\PYG@tc##1{\textcolor[rgb]{0.13,0.50,0.31}{##1}}}
\expandafter\def\csname PYG@tok@gp\endcsname{\let\PYG@bf=\textbf\def\PYG@tc##1{\textcolor[rgb]{0.78,0.36,0.04}{##1}}}
\expandafter\def\csname PYG@tok@sh\endcsname{\def\PYG@tc##1{\textcolor[rgb]{0.25,0.44,0.63}{##1}}}
\expandafter\def\csname PYG@tok@ow\endcsname{\let\PYG@bf=\textbf\def\PYG@tc##1{\textcolor[rgb]{0.00,0.44,0.13}{##1}}}
\expandafter\def\csname PYG@tok@sx\endcsname{\def\PYG@tc##1{\textcolor[rgb]{0.78,0.36,0.04}{##1}}}
\expandafter\def\csname PYG@tok@bp\endcsname{\def\PYG@tc##1{\textcolor[rgb]{0.00,0.44,0.13}{##1}}}
\expandafter\def\csname PYG@tok@c1\endcsname{\let\PYG@it=\textit\def\PYG@tc##1{\textcolor[rgb]{0.25,0.50,0.56}{##1}}}
\expandafter\def\csname PYG@tok@o\endcsname{\def\PYG@tc##1{\textcolor[rgb]{0.40,0.40,0.40}{##1}}}
\expandafter\def\csname PYG@tok@kc\endcsname{\let\PYG@bf=\textbf\def\PYG@tc##1{\textcolor[rgb]{0.00,0.44,0.13}{##1}}}
\expandafter\def\csname PYG@tok@c\endcsname{\let\PYG@it=\textit\def\PYG@tc##1{\textcolor[rgb]{0.25,0.50,0.56}{##1}}}
\expandafter\def\csname PYG@tok@mf\endcsname{\def\PYG@tc##1{\textcolor[rgb]{0.13,0.50,0.31}{##1}}}
\expandafter\def\csname PYG@tok@err\endcsname{\def\PYG@bc##1{\setlength{\fboxsep}{0pt}\fcolorbox[rgb]{1.00,0.00,0.00}{1,1,1}{\strut ##1}}}
\expandafter\def\csname PYG@tok@mb\endcsname{\def\PYG@tc##1{\textcolor[rgb]{0.13,0.50,0.31}{##1}}}
\expandafter\def\csname PYG@tok@ss\endcsname{\def\PYG@tc##1{\textcolor[rgb]{0.32,0.47,0.09}{##1}}}
\expandafter\def\csname PYG@tok@sr\endcsname{\def\PYG@tc##1{\textcolor[rgb]{0.14,0.33,0.53}{##1}}}
\expandafter\def\csname PYG@tok@mo\endcsname{\def\PYG@tc##1{\textcolor[rgb]{0.13,0.50,0.31}{##1}}}
\expandafter\def\csname PYG@tok@kd\endcsname{\let\PYG@bf=\textbf\def\PYG@tc##1{\textcolor[rgb]{0.00,0.44,0.13}{##1}}}
\expandafter\def\csname PYG@tok@mi\endcsname{\def\PYG@tc##1{\textcolor[rgb]{0.13,0.50,0.31}{##1}}}
\expandafter\def\csname PYG@tok@kn\endcsname{\let\PYG@bf=\textbf\def\PYG@tc##1{\textcolor[rgb]{0.00,0.44,0.13}{##1}}}
\expandafter\def\csname PYG@tok@cpf\endcsname{\let\PYG@it=\textit\def\PYG@tc##1{\textcolor[rgb]{0.25,0.50,0.56}{##1}}}
\expandafter\def\csname PYG@tok@kr\endcsname{\let\PYG@bf=\textbf\def\PYG@tc##1{\textcolor[rgb]{0.00,0.44,0.13}{##1}}}
\expandafter\def\csname PYG@tok@s\endcsname{\def\PYG@tc##1{\textcolor[rgb]{0.25,0.44,0.63}{##1}}}
\expandafter\def\csname PYG@tok@kp\endcsname{\def\PYG@tc##1{\textcolor[rgb]{0.00,0.44,0.13}{##1}}}
\expandafter\def\csname PYG@tok@w\endcsname{\def\PYG@tc##1{\textcolor[rgb]{0.73,0.73,0.73}{##1}}}
\expandafter\def\csname PYG@tok@kt\endcsname{\def\PYG@tc##1{\textcolor[rgb]{0.56,0.13,0.00}{##1}}}
\expandafter\def\csname PYG@tok@sc\endcsname{\def\PYG@tc##1{\textcolor[rgb]{0.25,0.44,0.63}{##1}}}
\expandafter\def\csname PYG@tok@sb\endcsname{\def\PYG@tc##1{\textcolor[rgb]{0.25,0.44,0.63}{##1}}}
\expandafter\def\csname PYG@tok@k\endcsname{\let\PYG@bf=\textbf\def\PYG@tc##1{\textcolor[rgb]{0.00,0.44,0.13}{##1}}}
\expandafter\def\csname PYG@tok@se\endcsname{\let\PYG@bf=\textbf\def\PYG@tc##1{\textcolor[rgb]{0.25,0.44,0.63}{##1}}}
\expandafter\def\csname PYG@tok@sd\endcsname{\let\PYG@it=\textit\def\PYG@tc##1{\textcolor[rgb]{0.25,0.44,0.63}{##1}}}

\def\PYGZbs{\char`\\}
\def\PYGZus{\char`\_}
\def\PYGZob{\char`\{}
\def\PYGZcb{\char`\}}
\def\PYGZca{\char`\^}
\def\PYGZam{\char`\&}
\def\PYGZlt{\char`\<}
\def\PYGZgt{\char`\>}
\def\PYGZsh{\char`\#}
\def\PYGZpc{\char`\%}
\def\PYGZdl{\char`\$}
\def\PYGZhy{\char`\-}
\def\PYGZsq{\char`\'}
\def\PYGZdq{\char`\"}
\def\PYGZti{\char`\~}
% for compatibility with earlier versions
\def\PYGZat{@}
\def\PYGZlb{[}
\def\PYGZrb{]}
\makeatother

\renewcommand\PYGZsq{\textquotesingle}

\begin{document}

\maketitle
\tableofcontents
\phantomsection\label{index::doc}

\begin{quote}\begin{description}
\item[{Author}] \leavevmode
Ronak H Shah

\item[{Contact}] \leavevmode
\href{mailto:rons.shah@gmail.com}{rons.shah@gmail.com}

\item[{Source code}] \leavevmode
\href{http://github.com/rhshah/iCallSV}{http://github.com/rhshah/iCallSV}

\item[{License}] \leavevmode
\href{http://www.apache.org/licenses/LICENSE-2.0}{Apache License 2.0}

\end{description}\end{quote}
\href{https://landscape.io/github/rhshah/iCallSV/master}{}
iCallSV is a Python library and command-line software toolkit to call structural aberrations from Next Generation DNA sequencing data. Behind the scenes it uses Delly2 to do structural variant calling. It is designed for use with hybrid capture, including both whole-exome and custom target panels, and
short-read sequencing platforms such as Illumina.


\chapter{Citation}
\label{index:citation}\label{index:icallsv-structural-aberration-detection-from-ngs-datasets}
We are in the process of publishing a manuscript describing iCallSV as part of the Structural Variant Detection framework.
If you use this software in a publication, for now, please cite our website \href{http://github.com/rhshah/iCallSV}{iCallSV}.

Contents:


\section{iCallSV: Structural Aberration Detection from NGS datasets}
\label{iCallSV::doc}\label{iCallSV:icallsv-structural-aberration-detection-from-ngs-datasets}\begin{quote}\begin{description}
\item[{Author}] \leavevmode
Ronak H Shah

\item[{Contact}] \leavevmode
\href{mailto:rons.shah@gmail.com}{rons.shah@gmail.com}

\item[{Source code}] \leavevmode
\href{http://github.com/rhshah/iCallSV}{http://github.com/rhshah/iCallSV}

\item[{Wiki}] \leavevmode
\href{http://icallsv.readthedocs.io/en/latest/}{http://icallsv.readthedocs.io/en/latest/}

\item[{License}] \leavevmode
\href{http://www.apache.org/licenses/LICENSE-2.0}{Apache License 2.0}

\end{description}\end{quote}
\href{https://landscape.io/github/rhshah/iCallSV/master}{}
iCallSV is a Python library and command-line software toolkit to call structural aberrations from Next Generation DNA sequencing data. Behind the scenes it uses Delly2 to do structural variant calling. It is designed for use with hybrid capture, including both whole-exome and custom target panels, and
short-read sequencing platforms such as Illumina.


\section{Citation}
\label{iCallSV:citation}
We are in the process of publishing a manuscript describing iCallSV as part of the Structural Variant Detection framework.
If you use this software in a publication, for now, please cite our website \href{http://github.com/rhshah/iCallSV}{iCallSV}.


\section{Required Packages}
\label{iCallSV:required-packages}
We require that you install:
\begin{quote}\begin{description}
\item[{pandas}] \leavevmode
\href{http://pandas.pydata.org/}{v0.16.2}

\item[{biopython}] \leavevmode
\href{http://biopython.org/wiki/Main\_Page}{v1.65}

\item[{pysam}] \leavevmode
\href{https://pypi.python.org/pypi/pysam}{v0.8.4}

\item[{pyvcf}] \leavevmode
\href{https://pypi.python.org/pypi/PyVCF}{0.6.7}

\item[{Delly}] \leavevmode
\href{https://github.com/tobiasrausch/delly}{v0.7.3}

\item[{targetSeqView}] \leavevmode
\href{https://github.com/Eitan177/targetSeqView}{master}

\item[{iAnnotateSV}] \leavevmode
\href{https://github.com/rhshah/iAnnotateSV/tree/1.0.5}{v1.0.5}

\end{description}\end{quote}


\section{Required Data Files}
\label{iCallSV:required-data-files}
This files are given in the \code{data} folder inside iCallSV, they are uploaded using \href{https://git-lfs.github.com/}{git-lfs} and need to be downloaded with \href{https://git-lfs.github.com/}{git-lfs}
\begin{quote}\begin{description}
\item[{blacklistRegionsFile}] \leavevmode
Tab-delimited file wihout header having black listed regions.
\begin{quote}\begin{description}
\item[{Example}] \leavevmode
7       140498077       5       175998094

\end{description}\end{quote}

\item[{blacklistGenes}] \leavevmode
Gene listed one per line wihout header that are to be removed
\begin{quote}\begin{description}
\item[{Example}] \leavevmode
LINC00486

CNOT4

\end{description}\end{quote}

\item[{genesToInclude}] \leavevmode
Gene listed one per line wihout header that are to be kept
\begin{quote}\begin{description}
\item[{Example}] \leavevmode
ALK

BRAF

\end{description}\end{quote}

\end{description}\end{quote}


\section{Configuration File Format}
\label{iCallSV:configuration-file-format}
\begin{Verbatim}[commandchars=\\\{\}]
\PYG{c+c1}{\PYGZsh{}\PYGZti{}\PYGZti{}\PYGZti{}Template configuration file to run iCallSV\PYGZti{}\PYGZti{}\PYGZti{}\PYGZsh{}}
\PYG{c+c1}{\PYGZsh{}\PYGZsh{}\PYGZsh{}\PYGZsh{} Path to python executable \PYGZsh{}\PYGZsh{}\PYGZsh{}}
\PYG{o}{[}Python\PYG{o}{]}
PYTHON:
\PYG{c+c1}{\PYGZsh{}\PYGZsh{}\PYGZsh{}\PYGZsh{} Path to R executable and R Lib \PYGZsh{}\PYGZsh{}\PYGZsh{}}
\PYG{o}{[}R\PYG{o}{]}
RHOME:
RLIB:
\PYG{c+c1}{\PYGZsh{}\PYGZsh{}\PYGZsh{}\PYGZsh{} Path to delly, bcftools executables and Version of delly (supports only 0.7.3)\PYGZsh{}\PYGZsh{}\PYGZsh{}}
\PYG{o}{[}SVcaller\PYG{o}{]}
DELLY:
DellyVersion:
BCFTOOLS:
\PYG{c+c1}{\PYGZsh{}\PYGZsh{}\PYGZsh{}\PYGZsh{} Path to hg19 Referece Fasta file \PYGZsh{}\PYGZsh{}\PYGZsh{}}
\PYG{o}{[}ReferenceFasta\PYG{o}{]}
REFFASTA:
\PYG{c+c1}{\PYGZsh{}\PYGZsh{}\PYGZsh{}\PYGZsh{} Path to file containing regions to exclude please follow Delly documentation for this \PYGZsh{}\PYGZsh{}\PYGZsh{}}
\PYG{o}{[}ExcludeRegion\PYG{o}{]}
EXREGIONS:
\PYG{c+c1}{\PYGZsh{}\PYGZsh{}\PYGZsh{}\PYGZsh{} Path to file containing regions to where lenient threshold will be used; and file containing genes to keep \PYGZsh{}\PYGZsh{}\PYGZsh{}}
\PYG{o}{[}HotSpotRegions\PYG{o}{]}
HotspotFile:
GenesToKeep:
\PYG{c+c1}{\PYGZsh{}\PYGZsh{}\PYGZsh{}\PYGZsh{} Path to file containing regions/genes to filter \PYGZsh{}\PYGZsh{}\PYGZsh{}}
\PYG{o}{[}BlackListRegions\PYG{o}{]}
BlackListFile:
BlackListGenes:
\PYG{c+c1}{\PYGZsh{}\PYGZsh{}\PYGZsh{}\PYGZsh{} Path to samtools executable \PYGZsh{}\PYGZsh{}\PYGZsh{}}
\PYG{o}{[}SAMTOOLS\PYG{o}{]}
SAMTOOLS:
\PYG{c+c1}{\PYGZsh{}\PYGZsh{}\PYGZsh{}\PYGZsh{} Path to iAnnotateSV.py and all its required files, please follow iAnnotateSV documentation \PYGZsh{}\PYGZsh{}\PYGZsh{}}
\PYG{o}{[}iAnnotateSV\PYG{o}{]}
ANNOSV:
GENOMEBUILD:
DISTANCE:
CANONICALTRANSCRIPTFILE:
UNIPROTFILE:
CosmicCensus:
RepeatRegionAnnotation:
DGvAnnotations:
\PYG{c+c1}{\PYGZsh{}\PYGZsh{}\PYGZsh{}\PYGZsh{} TargetSeqView Parameters \PYGZsh{}\PYGZsh{}\PYGZsh{}}
\PYG{o}{[}TargetSeqView\PYG{o}{]}
CalculateConfidenceScore:
GENOMEBUILD:
ReadLength:
\PYG{c+c1}{\PYGZsh{}\PYGZsh{}\PYGZsh{}\PYGZsh{} Parameters to run Delly \PYGZsh{}\PYGZsh{}\PYGZsh{}}
\PYG{o}{[}ParametersToRunDelly\PYG{o}{]}
MAPQ: 20
NumberOfProcessors: 4
\PYG{o}{[}ParametersToFilterDellyResults\PYG{o}{]}
\PYG{c+c1}{\PYGZsh{}\PYGZsh{}\PYGZsh{}\PYGZsh{}Case Allele Fraction Hotspot\PYGZsh{}\PYGZsh{}\PYGZsh{}\PYGZsh{}}
CaseAltFreqHotspot: 0.05
\PYG{c+c1}{\PYGZsh{}\PYGZsh{}\PYGZsh{}\PYGZsh{}Total Case Coverage Hotspot\PYGZsh{}\PYGZsh{}\PYGZsh{}\PYGZsh{}\PYGZsh{}}
\PYG{n+nv}{CaseCoverageHotspot} \PYG{o}{=} 5
\PYG{c+c1}{\PYGZsh{}\PYGZsh{}\PYGZsh{}\PYGZsh{}Control Allele Fraction Hotspot\PYGZsh{}\PYGZsh{}\PYGZsh{}\PYGZsh{}}
\PYG{n+nv}{ControlAltFreqHotspot} \PYG{o}{=} 0
\PYG{c+c1}{\PYGZsh{}\PYGZsh{}\PYGZsh{}\PYGZsh{}Case Allele Fraction\PYGZsh{}\PYGZsh{}\PYGZsh{}\PYGZsh{}}
CaseAltFreq: 0.08
\PYG{c+c1}{\PYGZsh{}\PYGZsh{}\PYGZsh{}\PYGZsh{}Total Case Coverage\PYGZsh{}\PYGZsh{}\PYGZsh{}\PYGZsh{}\PYGZsh{}}
\PYG{n+nv}{CaseCoverage} \PYG{o}{=} 8
\PYG{c+c1}{\PYGZsh{}\PYGZsh{}\PYGZsh{}\PYGZsh{}Control Allele Fraction\PYGZsh{}\PYGZsh{}\PYGZsh{}\PYGZsh{}}
\PYG{n+nv}{ControlAltFreq} \PYG{o}{=} 0
\PYG{c+c1}{\PYGZsh{}\PYGZsh{}\PYGZsh{}Overall Supporting Read\PYGZhy{}pairs \PYGZsh{}\PYGZsh{}\PYGZsh{}}
OverallSupportingReads: 5
\PYG{c+c1}{\PYGZsh{}\PYGZsh{}\PYGZsh{}Overall Supporting Read\PYGZhy{}pairs Hotspot \PYGZsh{}\PYGZsh{}\PYGZsh{}}
OverallSupportingReadsHotspot: 3
\PYG{c+c1}{\PYGZsh{}\PYGZsh{}\PYGZsh{}Overall Supporting splitreads \PYGZsh{}\PYGZsh{}\PYGZsh{}}
OverallSupportingSplitReads: 0
\PYG{c+c1}{\PYGZsh{}\PYGZsh{}\PYGZsh{}Overall Supporting splitreads Hotspot \PYGZsh{}\PYGZsh{}\PYGZsh{}}
OverallSupportingSplitReadsHotspot: 0
\PYG{c+c1}{\PYGZsh{}\PYGZsh{}\PYGZsh{}Case Supporting Read\PYGZhy{}pairs \PYGZsh{}\PYGZsh{}\PYGZsh{}}
CaseSupportingReads: 2
\PYG{c+c1}{\PYGZsh{}\PYGZsh{}\PYGZsh{}Case Supporting splitreads \PYGZsh{}\PYGZsh{}\PYGZsh{}}
CaseSupportingSplitReads: 0
\PYG{c+c1}{\PYGZsh{}\PYGZsh{}\PYGZsh{}Case Supporting Read\PYGZhy{}pairs Hotspot \PYGZsh{}\PYGZsh{}\PYGZsh{}}
CaseSupportingReadsHotspot: 1
\PYG{c+c1}{\PYGZsh{}\PYGZsh{}\PYGZsh{}Case Supporting splitreads Hotspot \PYGZsh{}\PYGZsh{}\PYGZsh{}}
CaseSupportingSplitReadsHotspot: 0
\PYG{c+c1}{\PYGZsh{}\PYGZsh{}\PYGZsh{}Control Supporting Read\PYGZhy{}pairs \PYGZsh{}\PYGZsh{}\PYGZsh{}}
ControlSupportingReads: 5
\PYG{c+c1}{\PYGZsh{}\PYGZsh{}\PYGZsh{}Control Supporting Read\PYGZhy{}pairs Hotspot \PYGZsh{}\PYGZsh{}\PYGZsh{}}
ControlSupportingReadsHotspot: 5
\PYG{c+c1}{\PYGZsh{}\PYGZsh{}\PYGZsh{}Control Supporting splitreads \PYGZsh{}\PYGZsh{}\PYGZsh{}}
ControlSupportingSplitReads: 5
\PYG{c+c1}{\PYGZsh{}\PYGZsh{}\PYGZsh{}Control Supporting splitreads Hotspot \PYGZsh{}\PYGZsh{}\PYGZsh{}}
ControlSupportingSplitReadsHotspot: 5
\PYG{c+c1}{\PYGZsh{}\PYGZsh{}\PYGZsh{}Length of Structural Variant\PYGZsh{}\PYGZsh{}\PYGZsh{}}
LengthOfSV: 500
\PYG{c+c1}{\PYGZsh{}\PYGZsh{}\PYGZsh{}Overall Mapping Quality Threshold\PYGZsh{}\PYGZsh{}\PYGZsh{}}
OverallMapq: 20
\PYG{c+c1}{\PYGZsh{}\PYGZsh{}\PYGZsh{}Overall Mapping Quality Threshold Hotspot\PYGZsh{}\PYGZsh{}\PYGZsh{}}
OverallMapqHotspot: 5
\end{Verbatim}


\section{Quick Usage}
\label{iCallSV:quick-usage}
\begin{Verbatim}[commandchars=\\\{\}]
python iCallSV.py \PYGZhy{}sc /path/to/template.ini \PYGZhy{}abam /path/to/casebamFile \PYGZhy{}bbam /path/to/controlbamFile \PYGZhy{}aId caseID \PYGZhy{}bId controlId \PYGZhy{}o /path/to/output/directory \PYGZhy{}op prefix\PYGZus{}for\PYGZus{}the\PYGZus{}output\PYGZus{}files
\end{Verbatim}

\begin{Verbatim}[commandchars=\\\{\}]
\PYGZgt{} python iCallSV.py \PYGZhy{}h

usage: iCallSV.py [\PYGZhy{}h] [\PYGZhy{}v] [\PYGZhy{}V] \PYGZhy{}sc config.ini \PYGZhy{}abam caseBAMFile.bam \PYGZhy{}bbam
                  controlBAMFile.bam \PYGZhy{}aId caseID \PYGZhy{}bId controlID \PYGZhy{}o
                  /somepath/output \PYGZhy{}op TumorID

iCallSV.iCallSV \PYGZhy{}\PYGZhy{} wrapper to run iCallSV package

  Created by Ronak H Shah on 2015\PYGZhy{}03\PYGZhy{}30.
  Copyright 2015\PYGZhy{}2016 Ronak H Shah. All rights reserved.

  Licensed under the Apache License 2.0
  http://www.apache.org/licenses/LICENSE\PYGZhy{}2.0

  Distributed on an \PYGZdq{}AS IS\PYGZdq{} basis without warranties
  or conditions of any kind, either express or implied.

USAGE

optional arguments:
  \PYGZhy{}h, \PYGZhy{}\PYGZhy{}help            show this help message and exit
  \PYGZhy{}v, \PYGZhy{}\PYGZhy{}verbose         set verbosity level [default: True]
  \PYGZhy{}V, \PYGZhy{}\PYGZhy{}version         show program\PYGZsq{}s version number and exit
  \PYGZhy{}sc config.ini, \PYGZhy{}\PYGZhy{}svConfig config.ini
                        Full path to the structural variant configuration
  \PYGZhy{}abam caseBAMFile.bam, \PYGZhy{}\PYGZhy{}caseBam caseBAMFile.bam
                        Full path to the case bam file
  \PYGZhy{}bbam controlBAMFile.bam, \PYGZhy{}\PYGZhy{}controlBam controlBAMFile.bam
                        Full path to the control bam file
  \PYGZhy{}aId caseID, \PYGZhy{}\PYGZhy{}caseId caseID
                        Id of the case to be analyzed, this will be the sub\PYGZhy{}
                        folder
  \PYGZhy{}bId controlID, \PYGZhy{}\PYGZhy{}controlId controlID
                        Id of the control to be used, this will be used for
                        filtering variants
  \PYGZhy{}o /somepath/output, \PYGZhy{}\PYGZhy{}outDir /somepath/output
                        Full Path to the output dir.
  \PYGZhy{}op TumorID, \PYGZhy{}\PYGZhy{}outPrefix TumorID
                        Id of the Tumor bam file which will be used as the
                        prefix for output files
\end{Verbatim}


\section{Submodules}
\label{iCallSV:submodules}

\subsection{iCallSV.FilterDellyCalls module}
\label{iCallSV:icallsv-filterdellycalls-module}\label{iCallSV:module-iCallSV.FilterDellyCalls}\index{iCallSV.FilterDellyCalls (module)}

\subsubsection{FilterDellyCalls}
\label{iCallSV:filterdellycalls}\begin{quote}\begin{description}
\item[{Description}] \leavevmode
This module will filter calls made by Delly which are in a VCF format

\end{description}\end{quote}
\index{GetCaseFlag() (in module iCallSV.FilterDellyCalls)}

\begin{fulllineitems}
\phantomsection\label{iCallSV:iCallSV.FilterDellyCalls.GetCaseFlag}\pysiglinewithargsret{\code{iCallSV.FilterDellyCalls.}\bfcode{GetCaseFlag}}{\emph{caseDR}, \emph{caseDV}, \emph{preciseFlag}, \emph{caseRR}, \emph{caseRV}, \emph{caseAltFreq}, \emph{caseTotalCount}}{}
This will \code{check if the case sample passes or not}
\begin{quote}\begin{description}
\item[{Parameters}] \leavevmode\begin{itemize}
\item {} 
\textbf{\texttt{caseDR}} (\emph{\texttt{int}}) -- int representing number of reference reads for case reported by delly

\item {} 
\textbf{\texttt{caseDV}} (\emph{\texttt{int}}) -- int representing number of variant reads for case reported by delly

\item {} 
\textbf{\texttt{preciseFlag}} (\emph{\texttt{str}}) -- str representing if an event is precise or imprecise

\item {} 
\textbf{\texttt{caseRR}} (\emph{\texttt{int}}) -- int representing number of split reference reads for case reported by delly

\item {} 
\textbf{\texttt{caseRV}} (\emph{\texttt{int}}) -- int representing number of split variant reads for case reported by delly

\item {} 
\textbf{\texttt{caseAltFreq}} (\emph{\texttt{float}}) -- float representing altratio threshold for case

\item {} 
\textbf{\texttt{caseTotalCount}} (\emph{\texttt{int}}) -- int repeseting readcount threshold for case

\end{itemize}

\item[{Returns}] \leavevmode
A boolean tag indicating True or False

\item[{Return type}] \leavevmode
bool

\end{description}\end{quote}

\end{fulllineitems}

\index{GetControlFlag() (in module iCallSV.FilterDellyCalls)}

\begin{fulllineitems}
\phantomsection\label{iCallSV:iCallSV.FilterDellyCalls.GetControlFlag}\pysiglinewithargsret{\code{iCallSV.FilterDellyCalls.}\bfcode{GetControlFlag}}{\emph{controlDR}, \emph{controlDV}, \emph{preciseFlag}, \emph{controlRR}, \emph{controlRV}, \emph{controlAltFreq}}{}
This will \code{check if the control sample passes or not}
\begin{quote}\begin{description}
\item[{Parameters}] \leavevmode\begin{itemize}
\item {} 
\textbf{\texttt{controlDR}} (\emph{\texttt{int}}) -- int representing number of reference reads for control reported by delly

\item {} 
\textbf{\texttt{controlDV}} (\emph{\texttt{int}}) -- int representing number of variant reads for control reported by delly

\item {} 
\textbf{\texttt{preciseFlag}} (\emph{\texttt{str}}) -- str representing if an event is precise or imprecise

\item {} 
\textbf{\texttt{controlRR}} (\emph{\texttt{int}}) -- int representing number of split reference reads for control reported by delly

\item {} 
\textbf{\texttt{controlRV}} (\emph{\texttt{int}}) -- int representing number of split variant reads for control reported by delly

\item {} 
\textbf{\texttt{controlAltFreq}} (\emph{\texttt{float}}) -- float representing altratio threshold for control

\end{itemize}

\item[{Returns}] \leavevmode
A boolean tag indicating True or False

\item[{Return type}] \leavevmode
bool

\end{description}\end{quote}

\end{fulllineitems}

\index{GetFilteredRecords() (in module iCallSV.FilterDellyCalls)}

\begin{fulllineitems}
\phantomsection\label{iCallSV:iCallSV.FilterDellyCalls.GetFilteredRecords}\pysiglinewithargsret{\code{iCallSV.FilterDellyCalls.}\bfcode{GetFilteredRecords}}{\emph{dellyVarialbles}, \emph{thresholdVariables}, \emph{hotspotDict}, \emph{blacklist}}{}
This will \code{Filter one record at a time}
\begin{quote}\begin{description}
\item[{Parameters}] \leavevmode\begin{itemize}
\item {} 
\textbf{\texttt{dellyVariables}} (\emph{\texttt{str}}) -- str having all delly variables separated by '',''

\item {} 
\textbf{\texttt{thresholdVariables}} (\emph{\texttt{str}}) -- str having all delly threshold variables separated by '',''

\item {} 
\textbf{\texttt{hotspotDict}} (\emph{\texttt{dict}}) -- A dict containing hotspot regions

\item {} 
\textbf{\texttt{blacklist}} (\emph{\texttt{list}}) -- A list containing blacklist regions

\end{itemize}

\item[{Returns}] \leavevmode
A boolean tag indicating True or False

\item[{Return type}] \leavevmode
bool

\end{description}\end{quote}

\end{fulllineitems}

\index{run() (in module iCallSV.FilterDellyCalls)}

\begin{fulllineitems}
\phantomsection\label{iCallSV:iCallSV.FilterDellyCalls.run}\pysiglinewithargsret{\code{iCallSV.FilterDellyCalls.}\bfcode{run}}{\emph{inputVcf}, \emph{outputDir}, \emph{controlId}, \emph{caseID}, \emph{hotspotFile}, \emph{blacklistFile}, \emph{svlength}, \emph{mapq}, \emph{mapqHotspot}, \emph{caseAltFreqHotspot}, \emph{caseTotalCountHotspot}, \emph{controlAltFreqHotspot}, \emph{caseAltFreq}, \emph{caseTotalCount}, \emph{controlAltFreq}, \emph{peSupport}, \emph{srSupport}, \emph{peSupportHotspot}, \emph{srSupportHotspot}, \emph{peSupportCase}, \emph{srSupportCase}, \emph{peSupportHotspotCase}, \emph{srSupportHotspotCase}, \emph{peSupportControl}, \emph{srSupportControl}, \emph{peSupportHotspotControl}, \emph{srSupportHotspotControl}, \emph{verbose}}{}
{\color{red}\bfseries{}{}`{}`}main:{\color{red}\bfseries{}{}`{}`}Filter calls made by Delly which are in a VCF format
\begin{quote}\begin{description}
\item[{Parameters}] \leavevmode\begin{itemize}
\item {} 
\textbf{\texttt{inputVcf}} (\emph{\texttt{str}}) -- Input VCF file name with path

\item {} 
\textbf{\texttt{outputDir}} (\emph{\texttt{str}}) -- Output directory

\item {} 
\textbf{\texttt{controlId}} (\emph{\texttt{str}}) -- Control Sample ID (Should be part of Sample Name in VCF)

\item {} 
\textbf{\texttt{caseID}} (\emph{\texttt{str}}) -- Case Sample ID (Should be part of Sample Name in VCF)

\item {} 
\textbf{\texttt{hospotFile}} (\emph{\texttt{str}}) -- List of Genes that have Hotspot Structural Variants (Tab-delimited Format without header:chr    start    end    geneName).

\item {} 
\textbf{\texttt{blacklistFile}} (\emph{\texttt{str}}) -- List of Genes that have blacklist of Structural Variants (Tab-delimited Format without header:chr    start1    chr2     start2; where chr1==chr2, end==start2).

\item {} 
\textbf{\texttt{caseAltFreq}} (\emph{\texttt{float}}) -- Alternate Allele Frequency threshold for case

\item {} 
\textbf{\texttt{caseTotalCount}} (\emph{\texttt{int}}) -- Total ReadCount threshold for case

\item {} 
\textbf{\texttt{ccontrolAltFreq}} (\emph{\texttt{flaot}}) -- Alternate Allele Frequency threshold for control

\item {} 
\textbf{\texttt{peSupport}} (\emph{\texttt{int}}) -- overall pair-end read support threshold for the event

\item {} 
\textbf{\texttt{srSupport}} (\emph{\texttt{int}}) -- overall split-reads support threshold for the event

\item {} 
\textbf{\texttt{peSupportHotspot}} (\emph{\texttt{int}}) -- overall pair-end read support threshold for the event in hot-spot region

\item {} 
\textbf{\texttt{srSupportHotspot}} (\emph{\texttt{int}}) -- overall split-reads support threshold for the event in hot-spot region

\item {} 
\textbf{\texttt{peSupportCase}} (\emph{\texttt{int}}) -- pair-end read support threshold for the event in the Case sample

\item {} 
\textbf{\texttt{srSupportCase}} (\emph{\texttt{int}}) -- split-reads support threshold for the event in the Case sample

\item {} 
\textbf{\texttt{peSupportHotspotCase}} (\emph{\texttt{int}}) -- pair-end read support threshold for the event in hot-spot region for the Case sample

\item {} 
\textbf{\texttt{srSupportHotspotCase}} (\emph{\texttt{int}}) -- split-reads support threshold for the event in hot-spot region for the Case sample

\item {} 
\textbf{\texttt{peSupportControl}} (\emph{\texttt{int}}) -- pair-end read support threshold for the event in the Control sample

\item {} 
\textbf{\texttt{srSupportControl}} (\emph{\texttt{int}}) -- split-reads support threshold for the event in the Control sample

\item {} 
\textbf{\texttt{peSupportHotspotControl}} (\emph{\texttt{int}}) -- pair-end read support threshold for the event in hot-spot region for the Control sample

\item {} 
\textbf{\texttt{srSupportHotspotControl}} (\emph{\texttt{int}}) -- split-reads support threshold for the event in hot-spot region for the Control sample

\item {} 
\textbf{\texttt{svlength}} (\emph{\texttt{int}}) -- length of the structural variants

\item {} 
\textbf{\texttt{mapq}} (\emph{\texttt{int}}) -- overall mapping quality

\item {} 
\textbf{\texttt{mapqHotspot}} (\emph{\texttt{int}}) -- mapping quality for hot-spots

\end{itemize}

\item[{Returns}] \leavevmode
A str name of filtered vcf file

\item[{Return type}] \leavevmode
str

\end{description}\end{quote}

\end{fulllineitems}



\subsection{iCallSV.Run\_Delly module}
\label{iCallSV:module-iCallSV.Run_Delly}\label{iCallSV:icallsv-run-delly-module}\index{iCallSV.Run\_Delly (module)}

\subsubsection{Run\_Delly}
\label{iCallSV:run-delly}\begin{quote}\begin{description}
\item[{Description}] \leavevmode
Runs the delly program on case and control bam file to give its results

\end{description}\end{quote}
\index{run() (in module iCallSV.Run\_Delly)}

\begin{fulllineitems}
\phantomsection\label{iCallSV:iCallSV.Run_Delly.run}\pysiglinewithargsret{\code{iCallSV.Run\_Delly.}\bfcode{run}}{\emph{delly}, \emph{version}, \emph{bcftools}, \emph{analysisType}, \emph{reference}, \emph{controlBam}, \emph{caseBam}, \emph{caseId}, \emph{mapq}, \emph{excludeRegions}, \emph{outputdir}, \emph{verbose}, \emph{debug}}{}
This will Runs the delly program on case and control bam file to give its
results.
\begin{quote}\begin{description}
\item[{Parameters}] \leavevmode\begin{itemize}
\item {} 
\textbf{\texttt{delly}} (\emph{\texttt{str}}) -- Path to delly executables (0.7.3 or above)

\item {} 
\textbf{\texttt{bcftools}} (\emph{\texttt{str}}) -- Path to bcftools executables (1.3.1 or above)

\item {} 
\textbf{\texttt{type}} (\emph{\texttt{str}}) -- What ot run in delly, DEL:Deletion, DUP: Duplication,TRA:Translocation, INV:Inversion

\item {} 
\textbf{\texttt{reference}} (\emph{\texttt{str}}) -- Reference Genome that was used to align the reads.

\item {} 
\textbf{\texttt{controlBam}} (\emph{\texttt{str}}) -- Path to control/normal bam file

\item {} 
\textbf{\texttt{caseBam}} (\emph{\texttt{str}}) -- Path to case/tumor bam file

\item {} 
\textbf{\texttt{controlID}} (\emph{\texttt{str}}) -- Id of the control/normal sample

\item {} 
\textbf{\texttt{caseID}} (\emph{\texttt{str}}) -- Id of the case/tumor sample

\item {} 
\textbf{\texttt{mapq}} (\emph{\texttt{int}}) -- mapping quality cutoff for delly

\item {} 
\textbf{\texttt{excludeRegions}} (\emph{\texttt{str}}) -- Regions to be excluded for calling structural variation.

\item {} 
\textbf{\texttt{outputdir}} (\emph{\texttt{str}}) -- directory for the output of delly

\item {} 
\textbf{\texttt{debug}} (\emph{\texttt{bool}}) -- If you just wish to test what we will do

\end{itemize}

\item[{Returns}] \leavevmode
str of the output vcf

\item[{Return type}] \leavevmode
str

\end{description}\end{quote}

\end{fulllineitems}



\subsection{iCallSV.Run\_iAnnotateSV module}
\label{iCallSV:icallsv-run-iannotatesv-module}\label{iCallSV:module-iCallSV.Run_iAnnotateSV}\index{iCallSV.Run\_iAnnotateSV (module)}

\subsubsection{Run\_iAnnotate}
\label{iCallSV:run-iannotate}\begin{quote}\begin{description}
\item[{Description}] \leavevmode
This module will run iAnnotateSV package

\end{description}\end{quote}
\index{run() (in module iCallSV.Run\_iAnnotateSV)}

\begin{fulllineitems}
\phantomsection\label{iCallSV:iCallSV.Run_iAnnotateSV.run}\pysiglinewithargsret{\code{iCallSV.Run\_iAnnotateSV.}\bfcode{run}}{\emph{python}, \emph{iAnnotateSV}, \emph{build}, \emph{distance}, \emph{canonicalTranscriptFile}, \emph{uniprotFile}, \emph{cosmicFile}, \emph{repeatregionFile}, \emph{dgvFile}, \emph{inputTabFile}, \emph{outputPrefix}, \emph{outputDir}}{}
This module will run iAnnotateSV package.
\begin{quote}\begin{description}
\item[{Parameters}] \leavevmode\begin{itemize}
\item {} 
\textbf{\texttt{python}} (\emph{\texttt{str}}) -- Location for the python executable.

\item {} 
\textbf{\texttt{iAnnotateSV}} (\emph{\texttt{str}}) -- Location of the wrapper iAnnotateSV package (iAnnotateSV.py)

\item {} 
\textbf{\texttt{build}} (\emph{\texttt{str}}) -- Which human reference file to be used, hg18,hg19 or hg38

\item {} 
\textbf{\texttt{inputTabFile}} (\emph{\texttt{str}}) -- Tab-Delimited Input FIle compatible with iAnnotateSV package.

\item {} 
\textbf{\texttt{outputPrefix}} (\emph{\texttt{str}}) -- Prefix of the output files/DIR with Annotations and images

\item {} 
\textbf{\texttt{outputDir}} (\emph{\texttt{str}}) -- Name of the output directory where the outputPrefix will be written

\item {} 
\textbf{\texttt{uniprotFile}} (\emph{\texttt{str}}) -- Location for ucsc uniprot file

\item {} 
\textbf{\texttt{cosmicFile}} (\emph{\texttt{str}}) -- Location for cosmic census file

\item {} 
\textbf{\texttt{repeatregionFile}} (\emph{\texttt{str}}) -- Location for repeat region file

\item {} 
\textbf{\texttt{dgvFile}} (\emph{\texttt{str}}) -- Location for database of Genomic Variants file

\end{itemize}

\item[{Returns}] \leavevmode
str of the output file

\item[{Return type}] \leavevmode
str

\end{description}\end{quote}

\end{fulllineitems}



\subsection{iCallSV.Run\_samblaster module}
\label{iCallSV:icallsv-run-samblaster-module}\label{iCallSV:module-iCallSV.Run_samblaster}\index{iCallSV.Run\_samblaster (module)}

\subsubsection{Run\_samblaster}
\label{iCallSV:run-samblaster}
:Description : This module will run samblaster for extracting discordant and spit reads in sam format
\index{run() (in module iCallSV.Run\_samblaster)}

\begin{fulllineitems}
\phantomsection\label{iCallSV:iCallSV.Run_samblaster.run}\pysiglinewithargsret{\code{iCallSV.Run\_samblaster.}\bfcode{run}}{\emph{samtools}, \emph{samblaster}, \emph{bamFile}, \emph{discordantFileName}, \emph{splitFileName}, \emph{outputDir}}{}
\end{fulllineitems}



\subsection{iCallSV.Run\_targetSeqView module}
\label{iCallSV:module-iCallSV.Run_targetSeqView}\label{iCallSV:icallsv-run-targetseqview-module}\index{iCallSV.Run\_targetSeqView (module)}

\subsubsection{Run\_targetSeqView}
\label{iCallSV:run-targetseqview}\begin{quote}\begin{description}
\item[{Description}] \leavevmode
This module will run targetSeqView

\end{description}\end{quote}
\index{run() (in module iCallSV.Run\_targetSeqView)}

\begin{fulllineitems}
\phantomsection\label{iCallSV:iCallSV.Run_targetSeqView.run}\pysiglinewithargsret{\code{iCallSV.Run\_targetSeqView.}\bfcode{run}}{\emph{RLocation}, \emph{targetSeqView}, \emph{nodes}, \emph{bamFilePath}, \emph{svFile}, \emph{build}, \emph{readLength}, \emph{outputDir}, \emph{outsvFileName}}{}
This module will run targetSeqView.
\begin{quote}\begin{description}
\item[{Parameters}] \leavevmode\begin{itemize}
\item {} 
\textbf{\texttt{RLocation}} (\emph{\texttt{str}}) -- Location of the R executable (\textgreater{}3.1.2).

\item {} 
\textbf{\texttt{targetSeqView}} (\emph{\texttt{str}}) -- Location of R script that will run tragetSeqView

\item {} 
\textbf{\texttt{nodes}} (\emph{\texttt{int}}) -- Number of parallel nodes for running targetSeqView

\item {} 
\textbf{\texttt{bamFile}} (\emph{\texttt{str}}) -- Location of the bamFile which has the  structural variant events.

\item {} 
\textbf{\texttt{svFile}} (\emph{\texttt{str}}) -- targetSeqView compatible input structural variant file.

\item {} 
\textbf{\texttt{build}} (\emph{\texttt{str}}) -- Which human reference file to be used, hg18,hg19 or hg38

\item {} 
\textbf{\texttt{readLength}} (\emph{\texttt{int}}) -- Sequencing Read Length (101)

\item {} 
\textbf{\texttt{outputDir}} (\emph{\texttt{str}}) -- Directory for output files

\item {} 
\textbf{\texttt{outsvFile}} (\emph{\texttt{str}}) -- Name of the output structural variant file that has added confidence score to it.

\end{itemize}

\item[{Returns}] \leavevmode
str of the output file

\item[{Return type}] \leavevmode
str

\end{description}\end{quote}

\end{fulllineitems}



\subsection{iCallSV.checkBlackList module}
\label{iCallSV:icallsv-checkblacklist-module}\label{iCallSV:module-iCallSV.checkBlackList}\index{iCallSV.checkBlackList (module)}

\subsubsection{checkBlackList}
\label{iCallSV:checkblacklist}\begin{quote}\begin{description}
\item[{Description}] \leavevmode
This module will read the Black List file and tell if and event is blacklisted or not

\end{description}\end{quote}
\index{CheckIfItIsBlacklisted() (in module iCallSV.checkBlackList)}

\begin{fulllineitems}
\phantomsection\label{iCallSV:iCallSV.checkBlackList.CheckIfItIsBlacklisted}\pysiglinewithargsret{\code{iCallSV.checkBlackList.}\bfcode{CheckIfItIsBlacklisted}}{\emph{chr1}, \emph{start1}, \emph{chr2}, \emph{start2}, \emph{blacklist}, \emph{extention}}{}
Check if coordinate are present in the \code{blacklist region file}
\begin{quote}\begin{description}
\item[{Parameters}] \leavevmode\begin{itemize}
\item {} 
\textbf{\texttt{chr1}} (\emph{\texttt{str}}) -- str of the breakpoint in first chromosome

\item {} 
\textbf{\texttt{start1}} (\emph{\texttt{int}}) -- int of the start location of the breakpoint in first chromosome

\item {} 
\textbf{\texttt{chr2}} (\emph{\texttt{str}}) -- str of the breakpoint in second chromosome

\item {} 
\textbf{\texttt{start2}} (\emph{\texttt{int}}) -- int of the start location of the breakpoint in second chromosome

\item {} 
\textbf{\texttt{blacklist}} (\emph{\texttt{list}}) -- A list containing black listed regions

\item {} 
\textbf{\texttt{extension}} (\emph{\texttt{int}}) -- an value for search in positive and negative direction of the start1 and start2 location

\end{itemize}

\item[{Returns}] \leavevmode
A boolean tag indicating True or False

\item[{Return type}] \leavevmode
bool

\end{description}\end{quote}

\end{fulllineitems}

\index{ReadBlackListFile() (in module iCallSV.checkBlackList)}

\begin{fulllineitems}
\phantomsection\label{iCallSV:iCallSV.checkBlackList.ReadBlackListFile}\pysiglinewithargsret{\code{iCallSV.checkBlackList.}\bfcode{ReadBlackListFile}}{\emph{BlackListFile}}{}
Read the \code{blacklist region file}
\begin{quote}\begin{description}
\item[{Parameters}] \leavevmode
\textbf{\texttt{BlackListFile}} (\emph{\texttt{str}}) -- str of file to be read.

\item[{Returns}] \leavevmode
A list containing black listed regions.

\item[{Return type}] \leavevmode
list.

\end{description}\end{quote}

\end{fulllineitems}



\subsection{iCallSV.checkHotSpotList module}
\label{iCallSV:icallsv-checkhotspotlist-module}\label{iCallSV:module-iCallSV.checkHotSpotList}\index{iCallSV.checkHotSpotList (module)}

\subsubsection{checkHotSpotList}
\label{iCallSV:checkhotspotlist}\begin{quote}\begin{description}
\item[{Description}] \leavevmode
This module will read the hotspot file and tell if it is a hotspot or not

\end{description}\end{quote}
\index{CheckIfItIsHotspot() (in module iCallSV.checkHotSpotList)}

\begin{fulllineitems}
\phantomsection\label{iCallSV:iCallSV.checkHotSpotList.CheckIfItIsHotspot}\pysiglinewithargsret{\code{iCallSV.checkHotSpotList.}\bfcode{CheckIfItIsHotspot}}{\emph{chr1}, \emph{start1}, \emph{chr2}, \emph{start2}, \emph{hotspotDict}}{}
Check if coordinate are present in the \code{hotspot region file}
\begin{quote}\begin{description}
\item[{Parameters}] \leavevmode\begin{itemize}
\item {} 
\textbf{\texttt{chr1}} (\emph{\texttt{str}}) -- str of the breakpoint in first chromosome

\item {} 
\textbf{\texttt{start1}} (\emph{\texttt{int}}) -- int of the start location of the breakpoint in first chromosome

\item {} 
\textbf{\texttt{chr2}} (\emph{\texttt{str}}) -- str of the breakpoint in second chromosome

\item {} 
\textbf{\texttt{start2}} (\emph{\texttt{int}}) -- int of the start location of the breakpoint in second chromosome

\item {} 
\textbf{\texttt{hotspotDict}} (\emph{\texttt{dict}}) -- A dict containing hotspot regions

\end{itemize}

\item[{Returns}] \leavevmode
A boolean tag indicating True or False

\item[{Return type}] \leavevmode
bool

\end{description}\end{quote}

\end{fulllineitems}

\index{ReadHotSpotFile() (in module iCallSV.checkHotSpotList)}

\begin{fulllineitems}
\phantomsection\label{iCallSV:iCallSV.checkHotSpotList.ReadHotSpotFile}\pysiglinewithargsret{\code{iCallSV.checkHotSpotList.}\bfcode{ReadHotSpotFile}}{\emph{HotSpotFile}}{}
Read the \code{HotSpot region file}
\begin{quote}\begin{description}
\item[{Parameters}] \leavevmode
\textbf{\texttt{HotSpotFile}} (\emph{\texttt{str}}) -- str of file to be read.

\item[{Returns}] \leavevmode
A dict containing hotspot regions

\item[{Return type}] \leavevmode
dict

\end{description}\end{quote}

\end{fulllineitems}



\subsection{iCallSV.checkparameters module}
\label{iCallSV:icallsv-checkparameters-module}\label{iCallSV:module-iCallSV.checkparameters}\index{iCallSV.checkparameters (module)}

\subsubsection{checkparameters}
\label{iCallSV:checkparameters}\begin{quote}\begin{description}
\item[{Description}] \leavevmode
This modules checks the parameters for various type of inputs.

\end{description}\end{quote}
\index{checkDellyAnalysisType() (in module iCallSV.checkparameters)}

\begin{fulllineitems}
\phantomsection\label{iCallSV:iCallSV.checkparameters.checkDellyAnalysisType}\pysiglinewithargsret{\code{iCallSV.checkparameters.}\bfcode{checkDellyAnalysisType}}{\emph{varaibleToCheck}}{}
Check \emph{if the variable for Delly analysis exists or not{}`}
\begin{quote}\begin{description}
\item[{Parameters}] \leavevmode
\textbf{\texttt{variableToCheck}} (\emph{\texttt{str}}) -- check if str is DEL\textbar{}DUP\textbar{}INV\textbar{}TRA

\item[{Returns}] \leavevmode
None

\item[{Return type}] \leavevmode
None

\end{description}\end{quote}

\end{fulllineitems}

\index{checkDir() (in module iCallSV.checkparameters)}

\begin{fulllineitems}
\phantomsection\label{iCallSV:iCallSV.checkparameters.checkDir}\pysiglinewithargsret{\code{iCallSV.checkparameters.}\bfcode{checkDir}}{\emph{folderToCheck}}{}
Check \emph{if the folder exists or not{}`}

\code{str}.
\begin{quote}\begin{description}
\item[{Parameters}] \leavevmode
\textbf{\texttt{folderToCheck}} (\emph{\texttt{str}}) -- Name of the folder to be checked.

\item[{Returns}] \leavevmode
None

\item[{Return type}] \leavevmode
None

\end{description}\end{quote}

\end{fulllineitems}

\index{checkEmpty() (in module iCallSV.checkparameters)}

\begin{fulllineitems}
\phantomsection\label{iCallSV:iCallSV.checkparameters.checkEmpty}\pysiglinewithargsret{\code{iCallSV.checkparameters.}\bfcode{checkEmpty}}{\emph{variableToCheck}, \emph{variableName}}{}
Check \emph{if the variable is None or not{}`}
\begin{quote}\begin{description}
\item[{Parameters}] \leavevmode\begin{itemize}
\item {} 
\textbf{\texttt{variableToCheck}} (\emph{\texttt{str}}) -- check if str is None or not

\item {} 
\textbf{\texttt{variableName}} (\emph{\texttt{str}}) -- Name of the None object to be verified

\end{itemize}

\item[{Returns}] \leavevmode
None

\item[{Return type}] \leavevmode
None

\end{description}\end{quote}

\end{fulllineitems}

\index{checkFile() (in module iCallSV.checkparameters)}

\begin{fulllineitems}
\phantomsection\label{iCallSV:iCallSV.checkparameters.checkFile}\pysiglinewithargsret{\code{iCallSV.checkparameters.}\bfcode{checkFile}}{\emph{fileToCheck}}{}
Check \emph{if the file exists or not{}`}
\begin{quote}\begin{description}
\item[{Parameters}] \leavevmode
\textbf{\texttt{fileToCheck}} (\emph{\texttt{str}}) -- Name of the file to be checked.

\item[{Returns}] \leavevmode
None

\item[{Return type}] \leavevmode
None

\end{description}\end{quote}

\end{fulllineitems}

\index{checkInt() (in module iCallSV.checkparameters)}

\begin{fulllineitems}
\phantomsection\label{iCallSV:iCallSV.checkparameters.checkInt}\pysiglinewithargsret{\code{iCallSV.checkparameters.}\bfcode{checkInt}}{\emph{variableToCheck}, \emph{variableName}}{}
Check \emph{if the variable is int or not{}`}
\begin{quote}\begin{description}
\item[{Parameters}] \leavevmode\begin{itemize}
\item {} 
\textbf{\texttt{variableToCheck}} (\emph{\texttt{int}}) -- Check if it is int or not

\item {} 
\textbf{\texttt{variableName}} (\emph{\texttt{str}}) -- Name of the int object to be verified

\end{itemize}

\item[{Returns}] \leavevmode
None

\item[{Return type}] \leavevmode
None

\end{description}\end{quote}

\end{fulllineitems}



\subsection{iCallSV.combineVCF module}
\label{iCallSV:module-iCallSV.combineVCF}\label{iCallSV:icallsv-combinevcf-module}\index{iCallSV.combineVCF (module)}

\subsubsection{combineVCF}
\label{iCallSV:combinevcf}\begin{quote}\begin{description}
\item[{Description}] \leavevmode
This module will combine multiple vcf file with same headers

\end{description}\end{quote}
\index{run() (in module iCallSV.combineVCF)}

\begin{fulllineitems}
\phantomsection\label{iCallSV:iCallSV.combineVCF.run}\pysiglinewithargsret{\code{iCallSV.combineVCF.}\bfcode{run}}{\emph{vcfFiles}, \emph{combinedVCF}, \emph{verbose}}{}
This will \code{combine multiple vcf file with same headers}
\begin{quote}\begin{description}
\item[{Parameters}] \leavevmode\begin{itemize}
\item {} 
\textbf{\texttt{vcfFiles}} (\emph{\texttt{list}}) -- a list of .vcf files to be combined

\item {} 
\textbf{\texttt{combinedVCF}} (\emph{\texttt{str}}) -- str for the output of combined vcf files

\item {} 
\textbf{\texttt{verbose}} (\emph{\texttt{bool}}) -- a boolean

\end{itemize}

\item[{Returns}] \leavevmode
A str name of combined vcf file

\item[{Return type}] \leavevmode
str

\end{description}\end{quote}

\end{fulllineitems}



\subsection{iCallSV.dellyVcf2Tab module}
\label{iCallSV:module-iCallSV.dellyVcf2Tab}\label{iCallSV:icallsv-dellyvcf2tab-module}\index{iCallSV.dellyVcf2Tab (module)}

\subsubsection{dellyVcf2Tab}
\label{iCallSV:dellyvcf2tab}\begin{quote}\begin{description}
\item[{Description}] \leavevmode
This module converts the Delly Vcf file having tumor normal, to tab-delimited format for input to iAnnotateSV

\end{description}\end{quote}
\index{vcf2tab() (in module iCallSV.dellyVcf2Tab)}

\begin{fulllineitems}
\phantomsection\label{iCallSV:iCallSV.dellyVcf2Tab.vcf2tab}\pysiglinewithargsret{\code{iCallSV.dellyVcf2Tab.}\bfcode{vcf2tab}}{\emph{vcfFile}, \emph{outputDir}, \emph{verbose}}{}
This \code{converts} the Delly Vcf file having tumor normal, to tab-delimited format for input to iAnnotateSV
\begin{quote}\begin{description}
\item[{Parameters}] \leavevmode\begin{itemize}
\item {} 
\textbf{\texttt{vcfFile}} (\emph{\texttt{str}}) -- str of vcf file to be converted

\item {} 
\textbf{\texttt{outputDir}} (\emph{\texttt{str}}) -- str for the output directory

\item {} 
\textbf{\texttt{verbose}} (\emph{\texttt{bool}}) -- a boolean

\end{itemize}

\item[{Returns}] \leavevmode
A str name of tab-delimited file

\item[{Return type}] \leavevmode
str

\end{description}\end{quote}

\end{fulllineitems}



\subsection{iCallSV.dellyVcf2targetSeqView module}
\label{iCallSV:module-iCallSV.dellyVcf2targetSeqView}\label{iCallSV:icallsv-dellyvcf2targetseqview-module}\index{iCallSV.dellyVcf2targetSeqView (module)}

\subsubsection{dellyVcf2targetSeqView}
\label{iCallSV:dellyvcf2targetseqview}\begin{quote}\begin{description}
\item[{Description}] \leavevmode
Convert VCF to targetSeqView

\end{description}\end{quote}
\index{Convert2targetSeqView() (in module iCallSV.dellyVcf2targetSeqView)}

\begin{fulllineitems}
\phantomsection\label{iCallSV:iCallSV.dellyVcf2targetSeqView.Convert2targetSeqView}\pysiglinewithargsret{\code{iCallSV.dellyVcf2targetSeqView.}\bfcode{Convert2targetSeqView}}{\emph{sampleName}, \emph{sampleBamName}, \emph{sampleSplitBamName}, \emph{vcfFile}, \emph{outputDir}, \emph{outputFileName}}{}
This \code{converts} the Delly Vcf file having tumor normal, to tab-delimited format for input to targetSeqView
\begin{quote}\begin{description}
\item[{Parameters}] \leavevmode\begin{itemize}
\item {} 
\textbf{\texttt{sampleName}} (\emph{\texttt{str}}) -- str for the name of the sample being analyzed

\item {} 
\textbf{\texttt{sampleBamName}} (\emph{\texttt{str}}) -- str for the pair-end reads bam file

\item {} 
\textbf{\texttt{sampleSplitBamName}} (\emph{\texttt{str}}) -- str for the split reads bam file

\item {} 
\textbf{\texttt{vcfFile}} (\emph{\texttt{str}}) -- str of vcf file to be converted

\item {} 
\textbf{\texttt{outputDir}} (\emph{\texttt{str}}) -- str for the output directory

\item {} 
\textbf{\texttt{outputFileName}} (\emph{\texttt{str}}) -- str for the output File

\end{itemize}

\item[{Returns}] \leavevmode
A str name of tab-delimited file

\item[{Return type}] \leavevmode
str

\end{description}\end{quote}

\end{fulllineitems}



\subsection{iCallSV.filterAnnotatedSV module}
\label{iCallSV:icallsv-filterannotatedsv-module}\label{iCallSV:module-iCallSV.filterAnnotatedSV}\index{iCallSV.filterAnnotatedSV (module)}

\subsubsection{filterAnnotatedSV}
\label{iCallSV:filterannotatedsv}\begin{quote}\begin{description}
\item[{Description}] \leavevmode
This module will filter calls from the merged file

\end{description}\end{quote}
\index{checkBlackListGene() (in module iCallSV.filterAnnotatedSV)}

\begin{fulllineitems}
\phantomsection\label{iCallSV:iCallSV.filterAnnotatedSV.checkBlackListGene}\pysiglinewithargsret{\code{iCallSV.filterAnnotatedSV.}\bfcode{checkBlackListGene}}{\emph{gene1}, \emph{gene2}, \emph{blacklistGenes}}{}
This will \code{check for blacklisted genes}
\begin{quote}\begin{description}
\item[{Parameters}] \leavevmode\begin{itemize}
\item {} 
\textbf{\texttt{gene1}} (\emph{\texttt{str}}) -- str for the name of gene at breakpoint 1

\item {} 
\textbf{\texttt{gene2}} (\emph{\texttt{str}}) -- str for the name of gene at breakpoint 2

\item {} 
\textbf{\texttt{blacklistGenes}} (\emph{\texttt{list}}) -- list containing blacklisted genes

\item {} 
\textbf{\texttt{genesToKeepFile}} (\emph{\texttt{str}}) -- str for the txt file containing genes to keep

\end{itemize}

\item[{Returns}] \leavevmode
A boolean tag indicating True or False

\item[{Return type}] \leavevmode
bool

\end{description}\end{quote}

\end{fulllineitems}

\index{checkEventInIntronFlag() (in module iCallSV.filterAnnotatedSV)}

\begin{fulllineitems}
\phantomsection\label{iCallSV:iCallSV.filterAnnotatedSV.checkEventInIntronFlag}\pysiglinewithargsret{\code{iCallSV.filterAnnotatedSV.}\bfcode{checkEventInIntronFlag}}{\emph{gene1}, \emph{gene2}, \emph{site1}, \emph{site2}}{}
This will \code{Check if the event is in the intron only and not affecting
splicing}
\begin{quote}\begin{description}
\item[{Parameters}] \leavevmode\begin{itemize}
\item {} 
\textbf{\texttt{gene1}} (\emph{\texttt{str}}) -- str for the name of gene at breakpoint 1

\item {} 
\textbf{\texttt{gene2}} (\emph{\texttt{str}}) -- str for the name of gene at breakpoint 2

\item {} 
\textbf{\texttt{site1}} (\emph{\texttt{str}}) -- str for the description of site in breakpoint 1

\item {} 
\textbf{\texttt{site2}} (\emph{\texttt{str}}) -- str for the description of site in breakpoint 2

\end{itemize}

\item[{Returns}] \leavevmode
A boolean tag indicating True or False

\item[{Return type}] \leavevmode
bool

\end{description}\end{quote}

\end{fulllineitems}

\index{checkGeneListToKeep() (in module iCallSV.filterAnnotatedSV)}

\begin{fulllineitems}
\phantomsection\label{iCallSV:iCallSV.filterAnnotatedSV.checkGeneListToKeep}\pysiglinewithargsret{\code{iCallSV.filterAnnotatedSV.}\bfcode{checkGeneListToKeep}}{\emph{gene1}, \emph{gene2}, \emph{keepGenes}}{}
\end{fulllineitems}

\index{run() (in module iCallSV.filterAnnotatedSV)}

\begin{fulllineitems}
\phantomsection\label{iCallSV:iCallSV.filterAnnotatedSV.run}\pysiglinewithargsret{\code{iCallSV.filterAnnotatedSV.}\bfcode{run}}{\emph{inputTxt}, \emph{outputDir}, \emph{outPrefix}, \emph{blacklistGenesFile}, \emph{genesToKeepFile}, \emph{verbose}}{}
This will \code{filter sv calls} from the final merged file.
\begin{quote}\begin{description}
\item[{Parameters}] \leavevmode\begin{itemize}
\item {} 
\textbf{\texttt{inputTxt}} (\emph{\texttt{str}}) -- str for the txt file to be filtered

\item {} 
\textbf{\texttt{outputDir}} (\emph{\texttt{str}}) -- str for the output directory

\item {} 
\textbf{\texttt{outputPrefix}} (\emph{\texttt{str}}) -- str prefix for the output File

\item {} 
\textbf{\texttt{blacklistGenesFile}} (\emph{\texttt{str}}) -- str for the txt file containing blacklisted genes

\item {} 
\textbf{\texttt{genesToKeepFile}} (\emph{\texttt{str}}) -- str for the txt file containing genes to keep

\item {} 
\textbf{\texttt{verbose}} (\emph{\texttt{bool}}) -- a boolean

\end{itemize}

\item[{Returns}] \leavevmode
A str name of final sv file

\item[{Return type}] \leavevmode
str

\end{description}\end{quote}

\end{fulllineitems}



\subsection{iCallSV.iCallSV module}
\label{iCallSV:icallsv-icallsv-module}\label{iCallSV:module-iCallSV.iCallSV}\index{iCallSV.iCallSV (module)}

\subsubsection{iCallSV}
\label{iCallSV:id7}\begin{quote}\begin{description}
\item[{Description}] \leavevmode
iCallSV is a wrapper to the iCallSV package which facilitates calling structural variants from Next Generation Sequencing methods such as Illumina

\item[{author}] \leavevmode
Ronak H Shah

\item[{copyright}] \leavevmode\begin{enumerate}
\setcounter{enumi}{2}
\item {} 
2015-2016 by Ronak H Shah for Memorial Sloan Kettering Cancer Center. All rights reserved.

\end{enumerate}

\item[{license}] \leavevmode
Apache License 2.0

\item[{contact}] \leavevmode
\href{mailto:rons.shah@gmail.com}{rons.shah@gmail.com}

\end{description}\end{quote}


\subsection{iCallSV.iCallSV\_dmp\_wrapper module}
\label{iCallSV:icallsv-icallsv-dmp-wrapper-module}\label{iCallSV:module-iCallSV.iCallSV_dmp_wrapper}\index{iCallSV.iCallSV\_dmp\_wrapper (module)}

\subsubsection{iCallSV\_dmp\_wrapper}
\label{iCallSV:icallsv-dmp-wrapper}\begin{quote}\begin{description}
\item[{Description}] \leavevmode
iCallSV is a wrapper to run the iCallSV package on MSKCC data

\item[{author}] \leavevmode
Ronak H Shah

\item[{copyright}] \leavevmode\begin{enumerate}
\setcounter{enumi}{2}
\item {} 
2015-2016 by Ronak H Shah for Memorial Sloan Kettering Cancer Center. All rights reserved.

\end{enumerate}

\item[{license}] \leavevmode
Apache License 2.0

\item[{contact}] \leavevmode
\href{mailto:rons.shah@gmail.com}{rons.shah@gmail.com}

\item[{deffield    updated}] \leavevmode
Updated

\end{description}\end{quote}
\index{RunJob() (in module iCallSV.iCallSV\_dmp\_wrapper)}

\begin{fulllineitems}
\phantomsection\label{iCallSV:iCallSV.iCallSV_dmp_wrapper.RunJob}\pysiglinewithargsret{\code{iCallSV.iCallSV\_dmp\_wrapper.}\bfcode{RunJob}}{\emph{cmd}}{}
Given a command run the job.
\begin{quote}\begin{description}
\item[{Parameters}] \leavevmode
\textbf{\texttt{cmd}} (\emph{\texttt{str}}) -- str of command to be run on the local machine

\item[{Returns}] \leavevmode
None

\item[{Return type}] \leavevmode
None

\end{description}\end{quote}

\end{fulllineitems}

\index{RunPerPool() (in module iCallSV.iCallSV\_dmp\_wrapper)}

\begin{fulllineitems}
\phantomsection\label{iCallSV:iCallSV.iCallSV_dmp_wrapper.RunPerPool}\pysiglinewithargsret{\code{iCallSV.iCallSV\_dmp\_wrapper.}\bfcode{RunPerPool}}{\emph{titleFile}, \emph{outdir}, \emph{HSmetricsFileList}, \emph{bamFileList}, \emph{args}}{}
This will run the pool to be analyzed.
\begin{quote}\begin{description}
\item[{Parameters}] \leavevmode\begin{itemize}
\item {} 
\textbf{\texttt{titleFile}} (\emph{\texttt{str}}) -- str of meta information file

\item {} 
\textbf{\texttt{outdir}} (\emph{\texttt{str}}) -- str of output directory

\item {} 
\textbf{\texttt{HSmetricsFileList}} (\emph{\texttt{list}}) -- list of picard hsmetrics files

\item {} 
\textbf{\texttt{bamFileList}} (\emph{\texttt{list}}) -- list of bam files

\item {} 
\textbf{\texttt{args}} (\emph{\texttt{Namespace}}) -- Namespace of args to get other variables

\end{itemize}

\item[{Returns}] \leavevmode
None

\item[{Return type}] \leavevmode
None

\end{description}\end{quote}

\end{fulllineitems}

\index{SelectNormal() (in module iCallSV.iCallSV\_dmp\_wrapper)}

\begin{fulllineitems}
\phantomsection\label{iCallSV:iCallSV.iCallSV_dmp_wrapper.SelectNormal}\pysiglinewithargsret{\code{iCallSV.iCallSV\_dmp\_wrapper.}\bfcode{SelectNormal}}{\emph{normal}, \emph{poolnormal}}{}
Select the best possible normal.
\begin{quote}\begin{description}
\item[{Parameters}] \leavevmode\begin{itemize}
\item {} 
\textbf{\texttt{normal}} (\emph{\texttt{str}}) -- str of match normal

\item {} 
\textbf{\texttt{poolnormal}} (\emph{\texttt{str}}) -- str of pool normal

\end{itemize}

\item[{Returns}] \leavevmode
str with decision whether to run matched or unmatched

\item[{Return type}] \leavevmode
str

\end{description}\end{quote}

\end{fulllineitems}

\index{SetupRun() (in module iCallSV.iCallSV\_dmp\_wrapper)}

\begin{fulllineitems}
\phantomsection\label{iCallSV:iCallSV.iCallSV_dmp_wrapper.SetupRun}\pysiglinewithargsret{\code{iCallSV.iCallSV\_dmp\_wrapper.}\bfcode{SetupRun}}{\emph{poolName}, \emph{args}}{}
This will setup the run to be analyzed.
\begin{quote}\begin{description}
\item[{Parameters}] \leavevmode\begin{itemize}
\item {} 
\textbf{\texttt{poolName}} (\emph{\texttt{str}}) -- str of pool to be analyzed

\item {} 
\textbf{\texttt{args}} (\emph{\texttt{Namespace}}) -- Namespace of args to get other variables

\end{itemize}

\item[{Returns}] \leavevmode
Multiple objects

\item[{Return type}] \leavevmode
list

\end{description}\end{quote}

\end{fulllineitems}

\index{getSubDirs() (in module iCallSV.iCallSV\_dmp\_wrapper)}

\begin{fulllineitems}
\phantomsection\label{iCallSV:iCallSV.iCallSV_dmp_wrapper.getSubDirs}\pysiglinewithargsret{\code{iCallSV.iCallSV\_dmp\_wrapper.}\bfcode{getSubDirs}}{\emph{dirLocation}}{}
Get all sub directories.
\begin{quote}\begin{description}
\item[{Parameters}] \leavevmode
\textbf{\texttt{dirLocation}} (\emph{\texttt{str}}) -- str of directory location

\item[{Returns}] \leavevmode
list of all sub directories

\item[{Return type}] \leavevmode
list

\end{description}\end{quote}

\end{fulllineitems}

\index{main() (in module iCallSV.iCallSV\_dmp\_wrapper)}

\begin{fulllineitems}
\phantomsection\label{iCallSV:iCallSV.iCallSV_dmp_wrapper.main}\pysiglinewithargsret{\code{iCallSV.iCallSV\_dmp\_wrapper.}\bfcode{main}}{}{}
\end{fulllineitems}

\index{processor() (in module iCallSV.iCallSV\_dmp\_wrapper)}

\begin{fulllineitems}
\phantomsection\label{iCallSV:iCallSV.iCallSV_dmp_wrapper.processor}\pysiglinewithargsret{\code{iCallSV.iCallSV\_dmp\_wrapper.}\bfcode{processor}}{\emph{i}, \emph{jobqueue}}{}
Operate on a jobqueue.
\begin{quote}\begin{description}
\item[{Parameters}] \leavevmode\begin{itemize}
\item {} 
\textbf{\texttt{i}} (\emph{\texttt{int}}) -- count of the job

\item {} 
\textbf{\texttt{jobqueue}} (\emph{\texttt{Namespace}}) -- Namespace for jobqueue

\end{itemize}

\item[{Returns}] \leavevmode
None

\item[{Return type}] \leavevmode
None

\end{description}\end{quote}

\end{fulllineitems}



\subsection{iCallSV.launchThreads module}
\label{iCallSV:icallsv-launchthreads-module}\label{iCallSV:module-iCallSV.launchThreads}\index{iCallSV.launchThreads (module)}
Created on December 21, 2015
Description: This module will be launching functions as threads
@author: Ronak H Shah
\index{myThread (class in iCallSV.launchThreads)}

\begin{fulllineitems}
\phantomsection\label{iCallSV:iCallSV.launchThreads.myThread}\pysiglinewithargsret{\strong{class }\code{iCallSV.launchThreads.}\bfcode{myThread}}{\emph{threadID}, \emph{name}, \emph{counter}}{}
Bases: \code{threading.Thread}
\index{run() (iCallSV.launchThreads.myThread method)}

\begin{fulllineitems}
\phantomsection\label{iCallSV:iCallSV.launchThreads.myThread.run}\pysiglinewithargsret{\bfcode{run}}{}{}
\end{fulllineitems}


\end{fulllineitems}

\index{print\_time() (in module iCallSV.launchThreads)}

\begin{fulllineitems}
\phantomsection\label{iCallSV:iCallSV.launchThreads.print_time}\pysiglinewithargsret{\code{iCallSV.launchThreads.}\bfcode{print\_time}}{\emph{threadName}, \emph{delay}, \emph{counter}}{}
\end{fulllineitems}



\subsection{iCallSV.launch\_FilterDellyCalls module}
\label{iCallSV:module-iCallSV.launch_FilterDellyCalls}\label{iCallSV:icallsv-launch-filterdellycalls-module}\index{iCallSV.launch\_FilterDellyCalls (module)}

\subsubsection{launch\_FilterDellyCalls}
\label{iCallSV:launch-filterdellycalls}\begin{quote}\begin{description}
\item[{Description}] \leavevmode
This module will filter delly results and create filtered delly vcf files

\end{description}\end{quote}
\index{launch\_filterdellycalls\_for\_different\_analysis\_type() (in module iCallSV.launch\_FilterDellyCalls)}

\begin{fulllineitems}
\phantomsection\label{iCallSV:iCallSV.launch_FilterDellyCalls.launch_filterdellycalls_for_different_analysis_type}\pysiglinewithargsret{\code{iCallSV.launch\_FilterDellyCalls.}\bfcode{launch\_filterdellycalls\_for\_different\_analysis\_type}}{\emph{args}, \emph{config}, \emph{sampleOutdirForDelly}, \emph{del\_vcf}, \emph{dup\_vcf}, \emph{inv\_vcf}, \emph{tra\_vcf}}{}
This will launch the filtering of delly calls in parallel.
\begin{quote}\begin{description}
\item[{Parameters}] \leavevmode\begin{itemize}
\item {} 
\textbf{\texttt{args}} (\emph{\texttt{Namespace}}) -- Namespace of args to get other variables

\item {} 
\textbf{\texttt{config}} (\emph{\texttt{Namespace}}) -- configuration file passed to iCallSV

\item {} 
\textbf{\texttt{sampleOutdirForDelly}} (\emph{\texttt{str}}) -- Output directory for delly vcf files.

\item {} 
\textbf{\texttt{del\_vcf}} (\emph{\texttt{str}}) -- Path to deletion based vcf file

\item {} 
\textbf{\texttt{dup\_vcf}} (\emph{\texttt{str}}) -- Path to duplication based vcf file

\item {} 
\textbf{\texttt{inv\_vcf}} (\emph{\texttt{str}}) -- Path to inversion based vcf file

\item {} 
\textbf{\texttt{tra\_vcf}} (\emph{\texttt{str}}) -- Path to translocation based vcf file

\end{itemize}

\item[{Returns}] \leavevmode
Multiple objects

\item[{Return type}] \leavevmode
list

\end{description}\end{quote}

\end{fulllineitems}



\subsection{iCallSV.launch\_Run\_Delly module}
\label{iCallSV:module-iCallSV.launch_Run_Delly}\label{iCallSV:icallsv-launch-run-delly-module}\index{iCallSV.launch\_Run\_Delly (module)}

\subsubsection{launch\_Run\_Delly}
\label{iCallSV:launch-run-delly}\begin{quote}\begin{description}
\item[{Description}] \leavevmode
This module will be launching delly using Run\_Delly

\end{description}\end{quote}
\index{launch\_delly\_for\_different\_analysis\_type() (in module iCallSV.launch\_Run\_Delly)}

\begin{fulllineitems}
\phantomsection\label{iCallSV:iCallSV.launch_Run_Delly.launch_delly_for_different_analysis_type}\pysiglinewithargsret{\code{iCallSV.launch\_Run\_Delly.}\bfcode{launch\_delly\_for\_different\_analysis\_type}}{\emph{args}, \emph{config}, \emph{sampleOutdirForDelly}}{}
This will launch delly calls in parallel.
\begin{quote}\begin{description}
\item[{Parameters}] \leavevmode\begin{itemize}
\item {} 
\textbf{\texttt{args}} (\emph{\texttt{Namespace}}) -- Namespace of args to get other variables

\item {} 
\textbf{\texttt{config}} (\emph{\texttt{Namespace}}) -- configuration file passed to iCallSV

\item {} 
\textbf{\texttt{sampleOutdirForDelly}} (\emph{\texttt{str}}) -- Output directory for delly vcf files.

\end{itemize}

\item[{Returns}] \leavevmode
Multiple objects

\item[{Return type}] \leavevmode
list

\end{description}\end{quote}

\end{fulllineitems}



\subsection{iCallSV.make\_analysis\_dir module}
\label{iCallSV:module-iCallSV.make_analysis_dir}\label{iCallSV:icallsv-make-analysis-dir-module}\index{iCallSV.make\_analysis\_dir (module)}

\subsubsection{make\_analysis\_dir}
\label{iCallSV:make-analysis-dir}\begin{quote}\begin{description}
\item[{Description}] \leavevmode
This module will make directory structure for running analysis

\end{description}\end{quote}
\index{makeOutputDir() (in module iCallSV.make\_analysis\_dir)}

\begin{fulllineitems}
\phantomsection\label{iCallSV:iCallSV.make_analysis_dir.makeOutputDir}\pysiglinewithargsret{\code{iCallSV.make\_analysis\_dir.}\bfcode{makeOutputDir}}{\emph{args}, \emph{tool}}{}
This will make the output directory tree.
\begin{quote}\begin{description}
\item[{Parameters}] \leavevmode
\textbf{\texttt{args}} (\emph{\texttt{Namespace}}) -- Namespace of args to get other variables

\item[{Returns}] \leavevmode
Multiple objects

\item[{Return type}] \leavevmode
list

\end{description}\end{quote}

\end{fulllineitems}



\subsection{iCallSV.makebamindex module}
\label{iCallSV:module-iCallSV.makebamindex}\label{iCallSV:icallsv-makebamindex-module}\index{iCallSV.makebamindex (module)}

\subsubsection{makebamindex}
\label{iCallSV:makebamindex}\begin{quote}\begin{description}
\item[{Description}] \leavevmode
Use PySAM to make bam index

\end{description}\end{quote}
\index{MakeIndex() (in module iCallSV.makebamindex)}

\begin{fulllineitems}
\phantomsection\label{iCallSV:iCallSV.makebamindex.MakeIndex}\pysiglinewithargsret{\code{iCallSV.makebamindex.}\bfcode{MakeIndex}}{\emph{bamFile}}{}
This will make bam index if not there for a bam file using pysam.
\begin{quote}\begin{description}
\item[{Parameters}] \leavevmode
\textbf{\texttt{bamFile}} (\emph{\texttt{str}}) -- Path to bam file

\item[{Returns}] \leavevmode
None

\item[{Return type}] \leavevmode
None

\end{description}\end{quote}

\end{fulllineitems}



\subsection{iCallSV.mergeFinalFiles module}
\label{iCallSV:module-iCallSV.mergeFinalFiles}\label{iCallSV:icallsv-mergefinalfiles-module}\index{iCallSV.mergeFinalFiles (module)}

\subsubsection{mergeFinalFiles}
\label{iCallSV:mergefinalfiles}\begin{quote}\begin{description}
\item[{Description}] \leavevmode
Merge VCF, iAnnotateSV tab and targetSeqView tab file into a single tab-delimited file

\end{description}\end{quote}
\index{run() (in module iCallSV.mergeFinalFiles)}

\begin{fulllineitems}
\phantomsection\label{iCallSV:iCallSV.mergeFinalFiles.run}\pysiglinewithargsret{\code{iCallSV.mergeFinalFiles.}\bfcode{run}}{\emph{aId}, \emph{bId}, \emph{vcfFile}, \emph{annoTab}, \emph{confTab}, \emph{outDir}, \emph{outputPrefix}, \emph{verbose}}{}
This will Merge VCF, iAnnotateSV tab and targetSeqView tab file into a single tab-delimited file
\begin{quote}\begin{description}
\item[{Parameters}] \leavevmode\begin{itemize}
\item {} 
\textbf{\texttt{aId}} (\emph{\texttt{str}}) -- Sample ID for case that has the structural abberations

\item {} 
\textbf{\texttt{bId}} (\emph{\texttt{str}}) -- Sample ID for control

\item {} 
\textbf{\texttt{vcfFile}} (\emph{\texttt{str}}) -- Delly filtered and merged VCF file

\item {} 
\textbf{\texttt{annoTab}} (\emph{\texttt{str}}) -- iAnnotateSV tab-delimited file with annotations

\item {} 
\textbf{\texttt{confTab}} (\emph{\texttt{str}}) -- targetSeqView tab-delimited file with probability score

\item {} 
\textbf{\texttt{outputDir}} (\emph{\texttt{str}}) -- Directory to write the output file

\item {} 
\textbf{\texttt{outputPrefix}} (\emph{\texttt{str}}) -- Output File Prefix

\end{itemize}

\item[{Returns}] \leavevmode
str of the tab-delimited file

\item[{Return type}] \leavevmode
str

\end{description}\end{quote}

\end{fulllineitems}



\subsection{iCallSV.sortbamByCoordinate module}
\label{iCallSV:module-iCallSV.sortbamByCoordinate}\label{iCallSV:icallsv-sortbambycoordinate-module}\index{iCallSV.sortbamByCoordinate (module)}

\subsubsection{sortbamByCoordinate}
\label{iCallSV:sortbambycoordinate}\begin{quote}\begin{description}
\item[{Description}] \leavevmode
This module will sort bam file by coordinate

\end{description}\end{quote}
\index{sortBam() (in module iCallSV.sortbamByCoordinate)}

\begin{fulllineitems}
\phantomsection\label{iCallSV:iCallSV.sortbamByCoordinate.sortBam}\pysiglinewithargsret{\code{iCallSV.sortbamByCoordinate.}\bfcode{sortBam}}{\emph{inputBam}, \emph{outputBamName}, \emph{outputDir}}{}
\end{fulllineitems}



\subsection{iCallSV.sortbamByReadName module}
\label{iCallSV:module-iCallSV.sortbamByReadName}\label{iCallSV:icallsv-sortbambyreadname-module}\index{iCallSV.sortbamByReadName (module)}

\subsubsection{sortbamByReadName}
\label{iCallSV:sortbambyreadname}\begin{quote}\begin{description}
\item[{Description}] \leavevmode
This module will sort bam file by name

\end{description}\end{quote}
\index{sortBam() (in module iCallSV.sortbamByReadName)}

\begin{fulllineitems}
\phantomsection\label{iCallSV:iCallSV.sortbamByReadName.sortBam}\pysiglinewithargsret{\code{iCallSV.sortbamByReadName.}\bfcode{sortBam}}{\emph{inputBam}, \emph{outputBamName}, \emph{outputDir}}{}
\end{fulllineitems}



\subsection{Module contents}
\label{iCallSV:module-contents}\label{iCallSV:module-iCallSV}\index{iCallSV (module)}

\chapter{Indices and tables}
\label{index:indices-and-tables}\begin{itemize}
\item {} 
\DUspan{xref,std,std-ref}{genindex}

\item {} 
\DUspan{xref,std,std-ref}{modindex}

\item {} 
\DUspan{xref,std,std-ref}{search}

\end{itemize}


\renewcommand{\indexname}{Python Module Index}
\begin{theindex}
\def\bigletter#1{{\Large\sffamily#1}\nopagebreak\vspace{1mm}}
\bigletter{i}
\item {\texttt{iCallSV}}, \pageref{iCallSV:module-iCallSV}
\item {\texttt{iCallSV.checkBlackList}}, \pageref{iCallSV:module-iCallSV.checkBlackList}
\item {\texttt{iCallSV.checkHotSpotList}}, \pageref{iCallSV:module-iCallSV.checkHotSpotList}
\item {\texttt{iCallSV.checkparameters}}, \pageref{iCallSV:module-iCallSV.checkparameters}
\item {\texttt{iCallSV.combineVCF}}, \pageref{iCallSV:module-iCallSV.combineVCF}
\item {\texttt{iCallSV.dellyVcf2Tab}}, \pageref{iCallSV:module-iCallSV.dellyVcf2Tab}
\item {\texttt{iCallSV.dellyVcf2targetSeqView}}, \pageref{iCallSV:module-iCallSV.dellyVcf2targetSeqView}
\item {\texttt{iCallSV.filterAnnotatedSV}}, \pageref{iCallSV:module-iCallSV.filterAnnotatedSV}
\item {\texttt{iCallSV.FilterDellyCalls}}, \pageref{iCallSV:module-iCallSV.FilterDellyCalls}
\item {\texttt{iCallSV.iCallSV}}, \pageref{iCallSV:module-iCallSV.iCallSV}
\item {\texttt{iCallSV.iCallSV\_dmp\_wrapper}}, \pageref{iCallSV:module-iCallSV.iCallSV_dmp_wrapper}
\item {\texttt{iCallSV.launch\_FilterDellyCalls}}, \pageref{iCallSV:module-iCallSV.launch_FilterDellyCalls}
\item {\texttt{iCallSV.launch\_Run\_Delly}}, \pageref{iCallSV:module-iCallSV.launch_Run_Delly}
\item {\texttt{iCallSV.launchThreads}}, \pageref{iCallSV:module-iCallSV.launchThreads}
\item {\texttt{iCallSV.make\_analysis\_dir}}, \pageref{iCallSV:module-iCallSV.make_analysis_dir}
\item {\texttt{iCallSV.makebamindex}}, \pageref{iCallSV:module-iCallSV.makebamindex}
\item {\texttt{iCallSV.mergeFinalFiles}}, \pageref{iCallSV:module-iCallSV.mergeFinalFiles}
\item {\texttt{iCallSV.Run\_Delly}}, \pageref{iCallSV:module-iCallSV.Run_Delly}
\item {\texttt{iCallSV.Run\_iAnnotateSV}}, \pageref{iCallSV:module-iCallSV.Run_iAnnotateSV}
\item {\texttt{iCallSV.Run\_samblaster}}, \pageref{iCallSV:module-iCallSV.Run_samblaster}
\item {\texttt{iCallSV.Run\_targetSeqView}}, \pageref{iCallSV:module-iCallSV.Run_targetSeqView}
\item {\texttt{iCallSV.sortbamByCoordinate}}, \pageref{iCallSV:module-iCallSV.sortbamByCoordinate}
\item {\texttt{iCallSV.sortbamByReadName}}, \pageref{iCallSV:module-iCallSV.sortbamByReadName}
\end{theindex}

\renewcommand{\indexname}{Index}
\printindex
\end{document}
